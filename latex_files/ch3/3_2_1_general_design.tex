\tikzstyle{startstop} = [rectangle, rounded corners, minimum width=3cm, minimum height=1cm,text centered, draw=black, fill=red!30]
\tikzstyle{io} = [trapezium, trapezium left angle=70, trapezium right angle=110, minimum width=3cm, minimum height=1cm, text centered, draw=black, fill=blue!30]
\tikzstyle{train_process} = [rectangle, minimum width=3cm, minimum height=.7cm, text centered, draw=black, fill=green!40]
\tikzstyle{infer_process} = [rectangle, minimum width=3cm, minimum height=.7cm, text centered, draw=black, fill=blue!30]

\tikzstyle{decision} = [diamond, minimum width=.1cm, minimum height=.1cm, text centered, draw=black, fill=green!30]

\tikzstyle{public_func_rev} = [draw=black, rotate=90, anchor=north, fill=blue!20, rounded corners]
\tikzstyle{public_func} = [draw=black, fill=blue!20, rounded corners]
\tikzstyle{arrow} = [thick,->,>=stealth]


\begin{figure}[h]
  \begin{center}
  %   \resizebox{.32\textwidth}{!}{
  %     \begin{tikzpicture}[node distance=1.4cm, scale=.1]
  %     \end{tikzpicture}
  %   }
    \resizebox{.8\textwidth}{!}{    
      \begin{tikzpicture}[node distance=1.2cm, scale=.1]
        % first graph
        % private functions
        \node (n1) [train_process, xshift=-.8cm] { $\_sample\_nuisance()$  };
        \node (n2) [train_process, below of=n1] { $\_define\_objectives()$  };
        \node (n3) [decision, below of=n2, yshift=-.5cm] { $grads$  };
        \node (n4) [train_process, left of=n3, yshift=-1cm, xshift=-1.5cm] { $\_solve\_gradients()$  };
        \node (n5) [train_process, right of=n3, yshift=-1cm, xshift=1.5cm] { $\_solve\_bo()$  };
        \node (n6) [train_process, below of=n3, yshift=-1.5cm] { $\_filter\_solutions()$  };
        \node (n7) [train_process, below of=n6] { $\_build\_boxes()$  };
        \node (n8) [decision, below of=n7] { $h?$  };
        \node (n9) [train_process, right of=n8, yshift=-.5cm, xshift=1cm] { $\_fit\_models()$  };

        \node (n10) [train_process, below of=n8, yshift=-1cm] { $\_define\_posterior()$  };

        % public functions
        \node (n11) [public_func_rev, right of=n3, yshift=-5.6cm, xshift=-0.3cm, minimum width=4.7cm] {$solve\_problems()$};
        \node (n12) [public_func_rev, right of=n9, yshift=-3.4cm, xshift=-.45cm, minimum width=4.7cm] {$estimate\_regions()$};
        \node (n13) [public_func_rev, right of=n5, yshift=-4.5cm, xshift=-2.3cm, minimum width=10.5cm] {$fit\_posterior()$};
        \node (n14) [public_func, left of=n3, yshift=0.3cm, xshift=-1.5cm] {$distance\_hist()$};
        \node (n15) [public_func, left of=n6, xshift=-2.2cm, yshift=-.3cm] {$compute\_eps()$};        
        \node (n16) [public_func, left of=n7, xshift=-1.8cm, yshift=-1.5cm] {$visualize\_region()$};

        % public functions inference
        \node (n17) [public_func, below of=n10, xshift=-2cm, yshift=-.5cm] {$sample()$};
        \node (n18) [public_func, below of=n17] {$compute\_expectation()$};        
        \node (n19) [public_func, below of=n10, xshift=4cm, yshift=-.5cm] {$eval\_unnorm\_posterior()$};
        \node (n20) [public_func, below of=n19] {$eval\_posterior()$};

        % public functions evaluation
        \node (n21) [public_func, below of=n18, yshift=-.7cm] {$compute\_ess()$};
        \node (n22) [public_func, below of=n20, yshift=-.7cm] {$compute\_divergence()$};        

        % add headers
        \node (algorithm) [below of=n1, yshift=2.5cm, xshift=1cm, minimum width=4cm, minimum height=1cm] {\huge{Implementation Design} };
        
        % backgrounds
        \draw [ultra thick, draw=black, fill=yellow, opacity=0.15, rounded corners=20pt] (-115,8) rectangle (70, -108);

        \draw [ultra thick, draw=black, fill=red, opacity=0.15, rounded corners=20pt] (-115,8) rectangle (57, -48);
        \draw [ultra thick, draw=black, fill=red, opacity=0.15, rounded corners=20pt] (-115,-50) rectangle (57, -108);


        \draw [ultra thick, draw=black, fill=green, opacity=0.15, rounded corners=20pt] (-115,-110) rectangle (70, -140);

        \draw [ultra thick, draw=black, fill=cyan, opacity=0.15, rounded corners=20pt] (-115,-142) rectangle (70, -160);

        \draw [ultra thick, draw=black, fill=magenta, opacity=0.05, rounded corners=20pt] (-115,18) rectangle (-64, -160);

        \draw [ultra thick, draw=black, fill=magenta, opacity=0.05, rounded corners=20pt] (-60,18) rectangle (70, -160);
        

        % arrows
        \draw [arrow] (n1) -- (n2);
        \draw [arrow] (n2) -- (n3);
        \draw [arrow] (n3) -- (n4);
        \draw [arrow] (n3) -- (n5);
        \draw [arrow] (n4) -- (n6);
        \draw [arrow] (n5) -- (n6);
        \draw [arrow] (n6) -- (n7);
        \draw [arrow] (n7) -- (n8);
        \draw [arrow] (n8) -- (n9);
        \draw [arrow] (n9) -- (n10);
        \draw [arrow] (n8) -- (n10);


        % second graph
        \node (pro1) [train_process, left of=n1, xshift=-7cm, yshift=-1cm, minimum width=4cm, minimum height=1cm] { define $d_i(\thetab) \forall i$  };
        \node (pro2) [train_process, below of=pro1, yshift=-1cm, minimum width=4cm, minimum height=1cm] { Solve $\thetab_i^*, d_i^*, \forall i$  };
        \node (filter) [train_process, below of=pro2, yshift=-1.5cm, minimum width=4cm, minimum height=1cm] { Filter solutions  };
        \node (proposal_region) [train_process, below of=filter, yshift=-0.15cm, minimum width=4cm, minimum height=1cm] {Construct $q_i \forall i$};
        \node (surrogate) [train_process, below of=proposal_region, yshift=-0.15cm, minimum width=4cm, minimum height=1cm] {Construct $\Tilde{d}_i \forall i$};
        \node (posterior) [train_process, below of=surrogate, yshift=-0.15cm, minimum width=4cm, minimum height=1cm] {Define $p_{d,\epsilon}(\thetab)$};
        
        \node (sample) [infer_process, below of=posterior, yshift=-1.2cm, minimum width=4cm, minimum height=1cm] {Draw $\{w_{ij}, \thetab_{ij} \}$ };


        % add headers
        \node (algorithm) [below of=pro1, yshift=3.5cm, minimum width=4cm, minimum height=1cm] {\huge{Algorithm} };
        
        \draw [arrow] (pro1) -- (pro2);
        \draw [arrow] (pro2) -- (filter);
        \draw [arrow] (filter) -- (proposal_region);
        \draw [arrow] (proposal_region) -- (surrogate);
        \draw [arrow] (surrogate) -- (posterior);
        \draw [arrow] (posterior) -- (sample);

        % % backgrounds
        % \draw [ultra thick, draw=black, fill=yellow, opacity=0.15, rounded corners=5pt] (-72,8) rectangle (-62,-108);

        % % \draw [ultra thick, draw=black, fill=red, opacity=0.15, rounded corners=20pt] (-60,8) rectangle (57, -48);
        % % \draw [ultra thick, draw=black, fill=red, opacity=0.15, rounded corners=20pt] (-60,-50) rectangle (57, -108);


        % % \draw [ultra thick, draw=black, fill=green, opacity=0.15, rounded corners=20pt] (-60,-110) rectangle (70, -140);

        % % \draw [ultra thick, draw=black, fill=cyan, opacity=0.15, rounded corners=20pt] (-60,-142) rectangle (70, -160);

        
        % \draw [arrow] (pro1) -- (pro2);
        % \draw [arrow] (pro2) -- (filter);
        % \draw [arrow] (filter) -- (proposal_region);
        % \draw [arrow] (proposal_region) -- (surrogate);

        
      \end{tikzpicture}
    }
  \end{center}
  \caption{Overview of the
ROMC implementation. The training part follows a sequential pattern;
the functions in the green ellipses must be called in a sequential
fashion for completing the training part and define the posterior
distribution. The functions in blue ellipses are the functionalities
provided to the user.}
\end{figure}


The overview of our implementation is illustrated in
figure~\ref{fig:romc_overview}; one may interpret
figure~\ref{fig:romc_overview} as a depiction of the main class and
the ellipses are the main routines of the class. Following
\proglang{Python}'s naming principles, the methods starting with an
underscore (green ellipses) represent internal (private) functions,
whereas the rest (blue ellipses) are the public methods that the user
interacts with. In figure~\ref{fig:romc_overview}, it can be easily
observed that the implementation follows consistently the algorithmic
view presented in~\ref{alg:romc_algorithm}. The training part includes
all the steps until the computation of the proposal regions
i.e. sampling the nuisance variables, defining the optimization
problems, solving them, constructing the regions and fitting local
surrogate models. The inference part comprises of evaluating the
unnormalized posterior (and the normalized when is possible), sampling
and computing an expectation. We also provide some utilities for
inspecting the training process, such as plotting the histogram of the
final distances or visualizing the constructed bounding
boxes. Finally, in the evaluation part, we provide two methods for
evaluating the inference; (a) computing the Effective Sample Size
(ESS) of the samples and (b) measuring the divergence between the
approximate posterior the ground-truth, if the latter is
available.\footnote{Normally, the ground-truth posterior is not
  available; However, this functionality is useful in cases where the
  posterior can be computed numerically or with an alternative method
  (i.e.\ ABC Rejection Sampling), and we would like to measure the
  discrepancy between the two approximations.}

\subsubsection*{Parallelizing ROMC method}

ROMC has the significant advantage of being fully parallelizable. We
exploit this fact by implementing a parallel version of all the major
components of the method; (a) solving the optimization problems, (b)
constructing bounding box regions, (c) sampling and (d) evaluating the
posterior. We parallelize these processes using the built-in
\proglang{Python} package \pkg{multiprocessing}. The specific package
enables concurrency, using subprocesses instead of threads, for
side-stepping the Global Interpreter (GIL). For activating the
parallel version of the algorithm, the user simpy has to pass the
argument \code{parallelize=True} when instantiating \code{ROMC}.

\begin{Code}
---------------------------------- python ----------------------------------
>>> elfi.ROMC(<output_node>, parallelize=True)
----------------------------------------------------------------------------
\end{Code}
  
\subsubsection*{Simple one-dimensional example}

For illustrating the functionalities we will use the following running
example introduced by~\cite{Ikonomov2019},

\begin{gather} \label{eq:1D_example}
  p(\theta) = \mathcal{U}(\theta;-2.5,2.5)\\ \label{eq:1D_example_eq_2}
  p(y|\theta) = \left\{
    \begin{array}{ll} \theta^4 + u & \mbox{if } \theta \in [-0.5, 0.5]
\\ |\theta| - c + u & \mbox{otherwise}
    \end{array} \right.\\ 
  u \sim \mathcal{N}(0,1)
\end{gather}

\noindent

The prior is a uniform distribution in the range $[-2.5, 2.5]$ and the
likelihood is defined at~\ref{eq:1D_example_eq_2}. There is only one
observation $y_0 = 0$. The inference in this particular example can be
performed quite easily without using a likelihood-free inference
approach. We can exploit this fact for validating the accuracy of our
implementation.

In the following code snippet, we code the model at \pkg{ELFI}.

\begin{Code}
------------------------------ python snippet ------------------------------
  import elfi import scipy.stats as ss
  import numpy as np

  def simulator(t1, batch_size=1,random_state=None):
    if t1 < -0.5:
        y = ss.norm(loc=-t1-c, scale=1).rvs(random_state=random_state)
    elif t1 <= 0.5:
        y = ss.norm(loc=t1**4, scale=1).rvs(random_state=random_state)
    else:
        y = ss.norm(loc=t1-c, scale=1).rvs(random_state=random_state)
    return y

  # observation
  y = 0
      
  # Elfi graph
  t1 = elfi.Prior('uniform', -2.5, 5)
  sim = elfi.Simulator(simulator, t1, observed=y)
  d = elfi.Distance('euclidean', sim)

  # Initialise the ROMC inference method
  bounds = [(-2.5, 2.5)] # limits of the prior
  parallelize = True # activate parallel execution
  romc = elfi.ROMC(d, bounds=bounds, parallelize=parallelize)
----------------------------------------------------------------------------
\end{Code}

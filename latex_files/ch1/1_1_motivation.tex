An implicit or simulator-based model is a parameterised stochastic
data generating mechanism \cite{Gutmann2016}. The key characteristic
of these models is that although we can sample (simulate) data points,
we cannot evaluate the likelihood of a specific set of observations
$\data$. Formally, a simulator-based model is described as a
parameterised family of probability density functions
$\{ p_{\yb|\thetab}(\yb) \}_{\thetab}$, whose closed-form is either
unknown or intractable to evaluate. Whereas evaluating
$p_{\yb|\thetab}(\yb)$ is intractable, sampling is
feasible. Practically, a simulator can be understood as a black-box
machine $M_r$\footnote{The subscript $r$ in $M_r$ indicates the
  \textit{random} simulator. In the next chapters we will introduce
  $M_d$ indicating the \textit{deterministic} simulator.} that
given a set of parameters $\thetab$, produces samples $\yb$ in a
stochastic manner i.e.\ $M_r(\thetab) \rightarrow \yb$.

Simulator-based models are particularly captivating due to the
modelling freedom they provide; any physical process that can be
conceptualised as a computer program of finite steps can be modelled
as a simulator-based model without any compromise. The modelling
freedom includes any amount of hidden (unobserved) internal variables
or rule-based decisions. Hence, implicit models are often used to
model physical phenomena in many fiels such as biology, neuroscience
etc. This degree of freedom comes at a cost; performing the inference
is particularly demanding from both the computational and the
mathematical perspective. The class of methods that try to approximate
the posterior distribution in these cases are called likelihood-free
inference methods. Unfortunately, the algorithms deployed so far
permit the inference only at low-dimensional parametric spaces,
i.e.\ $\thetab \in \mathbb{R}^D$ where $D$ is small\footnote{In broad terms, small refers to dimensionality $D$ less than a hundred.}.

\subsubsection*{\textit{Goal of implicit models}}

As in most Machine Learning (ML) concepts, the fundamental goal is the
derivation of one (or many) parameter configuration(s) $\thetab^*$
that \textit{describe} the data best i.e.\ generate
samples $M_r(\thetab^*)$ that are as close as possible to the observed
data $\data$. In our case, following the Bayesian approach, we treat
the parameters of interest $\thetab$ as random variables and we try to
\textit{infer} a posterior distribution $p(\thetab|\data)$ on them.

In simulator-based models it is infeasible to evaluate the posterior
$p(\thetab|\data) \propto L(\thetab)p(\thetab)$, due to the
intractability of the likelihood $L(\thetab) = p(\data|\thetab)$. The
following equation allows incorporating the simulator in the place of
the likelihood,

\begin{equation} \label{eq:likelihood}
  L(\thetab) =
  \lim_{\epsilon \to 0} c_\epsilon \int_{\yb \in B_{d,\epsilon}(\data)} p(\yb|\thetab)d\yb =
  \lim_{\epsilon \to 0} c_\epsilon \Pr(M_r(\thetab) \in B_{d,\epsilon}(\data))
\end{equation}
%
where $c_\epsilon$ is a proportionality factor dependend on $\epsilon$
and $B_{d,\epsilon}(\data)$ is a region around $\data$. Intuitively,
equation~\ref{eq:likelihood} describes that the likelihood of a
specific parameter configuration $\thetab$ is equal to the probability
that the simulator will produce outputs infinitively close to the
observations $\data$, using this configuration. For continuous cases,
the probability of perfectly replicating the observed data becomes
zero. In this scenario a relaxation must be introduced by accepting
simulated data that fall close to the data i.e. in a region
$B_{d,\epsilon}(\data)$ around $\data$, where $\epsilon > 0$. The
region can be defined as
$\region (\data) := \{\yb: d(\yb, \data) \leq \epsilon \}$ where
$d(\cdot, \cdot)$ can represent any valid distance. This relaxation
introduces the approximate likelihood~\eqref{eq:approx_likelihood} and
the approximate posterior~\eqref{eq:approx_posterior}.

\begin{equation} \label{eq:approx_likelihood}
  L_{d, \epsilon}(\thetab) = Pr(\yb \in \region(\data)), \text{where  } \epsilon > 0
\end{equation}

\begin{equation} \label{eq:approx_posterior}
  p_{d,\epsilon}(\thetab|\data) \propto L_{d, \epsilon}(\thetab) p(\thetab)
\end{equation}
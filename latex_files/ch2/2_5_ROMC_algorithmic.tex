Here we present ROMC as a sequence of algorithmic steps. The reader
may think of the algorithmic description as the backbone of the
implementation that will follow. At a high-level, ROMC can be split
into the training and the inference part.

\subsubsection*{Training part}
\noindent
At the training (fitting) part, the goal is the estimation of the
proposal regions $q_i$. The tasks are (a) sampling the nuisance
variables $\vb_i \sim p(\vb)$ (b) defining the optimisation problems
$\min_{\thetab} \: g_i(\thetab)$ (c) obtaining $\thetab_i^*$ (d)
checking whether $d_i^* \leq \epsilon$ and (e) building the bounding
box for obtaining the proposal region $q_i$. If gradients are
available, using a gradient-based method is advised for obtaining
$\thetab_i^*$ much faster. Providing $\nabla_{\thetab} g_i$ in
closed-form provides an upgrade in both accuracy and efficiency; If
closed-form description is not available, approximate gradients with
finite-differences requires two evaluations of $g_i$ for \emph{every}
parameter $\theta_d$, which can only work in low-dimensional
problems. When gradients are not available, the Bayesian optimisation
paradigm exists as an alternative solution. In this scenario, the
training part becomes slower due to fitting of the surrogate model and
the blind optimisation steps. Nevertheless, the subsequent task of
computing the proposal region $q_i$ becomes faster since $\hat{d}_i$
can be used instead of $g_i$. The algorithms~\ref{alg:training_GB}
and~\ref{alg:training_GP} present the above procedure.

\subsubsection*{Inference Part}
Performing the inference includes one or more of the following three
tasks; (a) sampling from the posterior
$ \thetab_i \sim p_{d, \epsilon}(\thetab|\data)$, (b) computing an
expectation $E_{\thetab|\data}[h(\thetab)]$ and (c) evaluating the
unnormalised posterior $p_{d, \epsilon}(\thetab|\data)$. Computing an
expectation can be done easily after weighted samples are obtained
using the equation~\ref{eq:expectation}, so we do not discuss it
separately.

\noindent
Sampling is performed by getting $n_2$ samples from each proposal
distribution $q_i$. For each sample $\thetab_{ij}$, the indicator
function is evaluated $\indicator{\regioni(\data)}(\thetab_{ij})$ for
checking if it lies inside the acceptance region. If so the
corresponding weight is computed as in \eqref{eq:sampling}. As before,
if a surrogate model $\hat{d}$ is available, it can be utilised for
evaluating the indicator function. At the sampling task, the
computational benefit of using the surrogate model is more valuable
compared to the evaluation of the posterior, because the indicator
function must be evaluated for a total of $n_1 \times n_2$ points.

\noindent
Evaluating the unnormalised posterior requires only the
deterministic functions $g_i$ and the prior distribution $p(\thetab)$;
there is no need for solving the optimisation problems and building
the proposal regions. The evaluation requires iterating over all $g_i$
and evaluating the distance from the observed data. In contrast, using
the GP approach, the optimisation part should be performed first for
fitting the surrogate models $\hat{d}_i(\thetab)$ and evaluate the
indicator function on them. This provides an important speed-up,
especially when running the simulator is computationally
expensive.

\noindent
In summary, we can state that the choice of using a Bayesian
optimisation approach provides a significant speed-up in the inference
part with the cost of making the training part slower and a possible
approximation error. It is typical in many Machine-Learning use cases,
being able to provide enough time and computational resources for the
training phase, but asking for efficiency in the inference
part.

\begin{minipage}{0.46\textwidth}
\begin{algorithm}[H]
    \centering
    \caption{Training Part - Gradient-based. Requires $g_i(\thetab), p(\thetab)$}\label{alg:training_GB}
    \begin{algorithmic}[1]
      \For{$i \gets 1 \textrm{ to } n$}
        \State Obtain $\thetab_i^*$ using a Gradient Optimiser
        \If{$g_i(\thetab_i^*) > \epsilon$}
        \State{go to} 1
        \Else
        \State Approximate $\jac_i^* = \nabla g_i(\theta)$ and $H_i \approx \jac^T_i\jac_i$
        \State Use Algorithm~\ref{alg:region_construction} to obtain $q_i$
        \EndIf      
      \EndFor
      \Return{$q_i, p(\theta), g_i(\theta)$}
    \end{algorithmic}
\end{algorithm}
\end{minipage}
\hfill
\begin{minipage}{0.46\textwidth}
\begin{algorithm}[H]
    \centering
    \caption{Training Part - Bayesian optimisation. Requires $g_i(\thetab), p(\thetab)$}\label{alg:training_GP}
    \begin{algorithmic}[1]
      \For{$i \gets 1 \textrm{ to } n$}
        \State Obtain $\thetab_i^*, \hat{d}_i(\thetab)$ using a GP approach
        \If{$g_i(\thetab_i^*) > \epsilon$}
        \State{go to} 1
        \Else
        \State Approximate $H_i \approx \jac^T_i \jac_i$
        \State Use Algorithm~\ref{alg:region_construction} to obtain $q_i$
        \EndIf      
      \EndFor
      \Return{$q_i, p(\theta), \hat{d}_i(\theta)$}
    \end{algorithmic}
\end{algorithm}
\end{minipage}

\begin{algorithm}[!ht]
	\caption{Computation of the proposal distribution $q_i$; Needs, a model of distance $d$, optimal point $\thetab_i^*$, number of refinements $K$, step size $\eta$ and curvature matrix $\hessian_i$ ($\jac_i^T\jac_i $ or GP Hessian)}\label{alg:region_construction}
	\begin{algorithmic}[1]
	\State Compute eigenvectors $\vb_{d}$ of $\hess_i$ {\scriptsize ($d = 1,\ldots,||\thetab ||)$}
	\For{$d \gets 1 \textrm{ to } ||\thetab||$}
		\State $\Tilde{\thetab} \gets \thetab_i^*$ \label{algstep:box_constr_start}
		\State $k \gets 0$
		\Repeat
        	\Repeat
                \State $\Tilde{\thetab} \gets \Tilde{\thetab} + \eta \ \vb_{d}$ \Comment{Large step size $\eta$.}
        	\Until{$d(f_i(\Tilde{\thetab}), \data) > \epsilon$}
        	\State $\Tilde{\thetab} \gets \Tilde{\thetab} - \eta \ \vb_{d}$
        	\State $\eta \gets \eta/2$ \Comment{More accurate region boundary}
        	\State $k \gets k + 1$
    	\Until $k = K$
    	\State Set final $\Tilde{\thetab}$ as region end point. \label{algstep:box_constr_end}
    	\State Repeat steps~\ref{algstep:box_constr_start}~-~\ref{algstep:box_constr_end} for $\mathbf{v}_{d} = - \mathbf{v}_{d}$
	\EndFor
	\State Fit a rectangular box around the region end points and define $q_i$ as uniform distribution
	\end{algorithmic}
\end{algorithm}

% \begin{minipage}{0.46\textwidth}
% \begin{algorithm}[H]
%     \centering
%     \caption{Evaluate unnormalised posterior - Gradient approach. Requires $g_i(\theta), p(\theta)$}\label{alg:posterior_GB}
%     \begin{algorithmic}[1]
%       \State $k \leftarrow 0$
%         \For {$i \gets 1 \textrm{ to } n_1$}
%           \If {$g_i(\theta) > \epsilon$}
%             \State $k \leftarrow k + 1$
%           \EndIf
%           \EndFor
%       \Return{$kp(\theta)$}
%     \end{algorithmic}
% \end{algorithm}
% \end{minipage}
% \hfill
% \begin{minipage}{0.46\textwidth}
% \begin{algorithm}[H]
%     \centering
%     \caption{Evaluate unnormalised posterior - GP approach. Requires $\hat{d}_i(\theta), p(\theta)$}\label{alg:posterior_GP}
%     \begin{algorithmic}[1]
%       \State $k \leftarrow 0$
%         \For {$i \gets 1 \textrm{ to } n_1$}
%           \If {$d_i(\theta) > \epsilon$}
%             \State $k \leftarrow k + 1$
%           \EndIf
%           \EndFor
%       \Return{$kp(\theta)$}
%     \end{algorithmic}
% \end{algorithm}
% \end{minipage}


% \begin{minipage}{0.46\textwidth}
\begin{algorithm}[H]
    \centering
    \caption{Sampling. Requires a function of distance $(g_i(\theta)$ or $\hat{d}_i$ or $\hat{g}_i), p(\theta), q_i$}\label{alg:sampling_GB}
    \begin{algorithmic}[1]
      \For {$i \gets 1 \textrm{ to } n_1$}
      \For {$j \gets 1 \textrm{ to } n_2$}
          \State $\thetab_{ij} \sim q_i$
          \If {$g_i(\thetab_{ij}) > \epsilon$}
            \State Reject $\theta_{ij}$
          \Else {}
            \State $w_{ij} = \frac{p(\thetab_{ij})}{q(\thetab_{ij})}$
            \State Accept $\thetab_{ij}$, with weight $w_{ij}$
          \EndIf
      \EndFor
      \EndFor
    \end{algorithmic}
\end{algorithm}
% \end{minipage}
% \hfill
% \begin{minipage}{0.46\textwidth}
% \begin{algorithm}[H]
%     \centering
%     \caption{Sampling - GP approach. Requires $\hat{d}_i(\theta), p(\theta), q_i$}\label{alg:sampling_GP}
%     \begin{algorithmic}[1]
%       \For {$i \gets 1 \textrm{ to } n_1$}
%       \For {$j \gets 1 \textrm{ to } n_2$}
%           \State $\theta_{ij} \sim q_i$
%           \If {$\hat{d}_i(\theta_{ij}) > \epsilon$}
%             \State Reject $\theta_{ij}$
%           \Else {}
%             \State $w_{ij} = \frac{p(\theta_{ij})}{q(\theta_{ij})}$
%             \State Accept $\theta_{ij}$, with weight $w_{ij}$
%           \EndIf
%       \EndFor
%       \EndFor
%     \end{algorithmic}
% \end{algorithm}
% \end{minipage}

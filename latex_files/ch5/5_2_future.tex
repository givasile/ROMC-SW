Future research directions involve both the theoretical and
implementation aspect of the ROMC approach.


In the theoretical part, an essential drawback of the ROMC approach is
the creation of a single bounding box for each objective
function. This simplification can be error-prone in situations where
we have ignorance about the properties of the objective
function. Consider, for example, the scenario where an objective
function defines multiple tiny disjoint acceptance areas; the
optimiser will (hopefully) reach only one, and a bounding box will be
constructed around it, compressing all the mass of the distribution
there. Therefore, this small region will erroneously dominate the
specific objective function. In general, the existence of multiple
disjoint acceptance regions complicates the inference. A naive
approach for improving this failure would be solving each optimisation
problem more than once, using different starting points. With this
approach, we hope that the optimiser will reach a different local
minimum in each execution. However, apart from the added computational
complexity, this approach does not guarantee an improvement. In
general, identifying all the acceptance regions in an unconstrained
problem with many local optimums is quite challenging.


Further research on the optimisation part, may also focus on modelling
the case where the parameter space is constrained. Consider, for
example, the case where a parameter represents a probability (as in
the tuberculosis example). In this case, the parameter is constrained
in the region $[0,1]$. Defining a prior with zero mass outside of this
area ensures that all non-acceptable values will not contribute to the
posterior. However, the optimiser does not consider the prior. Hence,
we can observe the phenomenon where the majority of the (or all)
acceptance regions is cancelled-out by the prior. In this case, the
posterior will be dominated by a few acceptance region parts that
overlap with the prior, which is not robust. Dealing with such
constrained non-linear optimisation problems, which is particularly
challenging, can serve as a second research line.


The previous two research directions overlap with the field of
optimisation. Another research avenue, more concentrated on the ROMC
method, is the approximation of the acceptance regions with a more
complex geometrical shape than the bounding box. In the current
implementation, the bounding box only serves as the proposal region
for sampling; running the indicator function afterwards, we manage to
discard the samples that are above the distance threshold. This
operation is quite expensive from a computational point of view. The
workaround proposed from the current implementation is fitting a local
surrogate model for avoiding running the expensive simulator. However,
fitting a surrogate model for every objective function is still
expensive. Approximating the acceptance region accurately with a more
complicated geometrical structure could remove the burden of running
the indicator function later. For example, a simple idea would be
approximating the acceptance region with a set of non-overlapping
bounding boxes. We insist on using the square as the basic shape
because we can efficiently sample from a uniform distribution defined
on it. A future research direction could investigate this capability.


On the implementation side, the improvements are more concrete. A critical
improvement would be performing the training and the inference phase
in a batched fashion. For example, the practitioner may indicate that
he would like to approximate the posterior in batches of N
optimisation problems; this means that the approximation should be
updated after each N new bounding boxes are constructed. Utilising
this feature, the user may interactively decide when he would like to
terminate the process. Finally, the implementation could also support
saving the state of the training or the inference process for
continuing later. In the current implementation, although the user can
save the inference result, he cannot interrupt the training phase and
continue later.
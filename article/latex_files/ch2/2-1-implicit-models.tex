An implicit or simulator-based model is a parameterised stochastic
data generating mechanism. The key characteristic of these models is
that we can sample data points, but we cannot evaluate the
likelihood. Formally, a simulator-based model is a parameterised
family of probability density functions
\(\{ p(\yb|\thetab) \}_{\thetab}\) whose closed-form is either unknown
or computationally intractable. Practically, a simulator
\( M_r(\thetab) \) can be understood as a random black-box machine
(computer code) that generates samples \(\yb\) in a stochastic manner
for any given a set of parameters \(\thetab\), i.e.
\( M_r(\thetab) \rightarrow \yb \). We can isolate the randomness by
introducing the stochastic nuisance variables \(\vb \sim p(\vb)\) so
that the simulator becomes a deterministic mapping \(\simulator\) that
maps \((\thetab, \vb)\) to the data \(\yb\), i.e.\
\(\yb=\simulator(\thetab,\vb)\).  In terms of computer code,
\(p(\vb)\) is defined as a random number generating process within the
computer code.  % Importantly, we can fix the value of \(\vb\) while
% changing \(\thetab\) by fixing the random seed used.

Simulator-based models provide considerable modelling freedom; any
physical process that can be conceptualised as a computer program of
finite steps can be modelled as a simulator-based model without any
compromise. The modelling freedom allows for any amount of hidden
(unobserved) internal variables or rule-based decisions. Hence,
implicit models are often used to model physical phenomena in the
natural sciences such as e.g.~genetics, epidemiology, or neuroscience. 
Further background on simulator-based models and
example applications can be found in the articles
by \citet{Gutmann2016, Lintusaari2017, Sisson2018, Cranmer2020}.


The modelling freedom of simulator-based models, however, comes at the
price of difficulties in inferring their parameters. Denoting observed
data by \(\data\), the main difficulty is that the likelihood function
\(L(\thetab) = p(\data|\thetab)\) is generally intractable. To better
see the sources of the intractability, and to address them, we go back
to the basic characterisation of the likelihood as the (rescaled)
probability of a parameter of the model to generate data \(\yb\) that
is similar to the observed data \(\data\). More formally, the
likelihood \(L(\thetab)\) equals
\begin{equation} \label{eq:likelihood}
  L(\thetab) = \lim_{\epsilon \to 0} c_\epsilon \int_{\yb \in B_{d,\epsilon}(\data)} p(\yb|\thetab)d\yb =
  \lim_{\epsilon \to 0} c_\epsilon \Pr(\simulator(\thetab, \vb) \in \region(\data)  \mid \theta)
\end{equation}
where \(c_\epsilon\) is a proportionality factor that depends on
\(\epsilon\) and \(\region(\data)\) is an \(\epsilon\) region around \(\data\)
that is defined via a distance function \(d\), i.e.\ \(\region(\data)
:= \{\yb: d(\yb, \data) \leq \epsilon \}\). 

The basic characterisation of the likelihood in \eqref{eq:likelihood}
highlights two sources of intractability: The first is the computation
of the probability
\(\Pr(\simulator(\thetab,\vb) \in \region(\data))\), the second is the
limit of \(\epsilon \to 0\).  Approximating the probability with
samples becomes computationally infeasible if \(\epsilon\) is too
small. Hence, a large class of inference methods work with
\(\epsilon > 0\), which leads to the approximate likelihood function
\(L_{d, \epsilon}(\thetab)\),
\begin{equation} \label{eq:approx_likelihood}
  L_{d, \epsilon}(\thetab) = \Pr(\yb \in \region(\data) \mid \theta), \quad \text{where  } \epsilon > 0.
\end{equation}

and, in turn, to the approximate posterior,

\begin{equation} \label{eq:approx_posterior}
  p_{d,\epsilon}(\thetab|\data) \propto L_{d, \epsilon}(\thetab) p(\thetab).
\end{equation}

The approximation in \eqref{eq:approx_likelihood} is by no means the
only strategy to deal with the intractabilities of the likelihood
function in \eqref{eq:likelihood}. Other strategies include modelling
the (stochastic) relationship between \(\thetab\) and \(\yb\), and its
reverse, or framing likelihood-free inference as a ratio estimation
problem, see for example \citet{Blum2010, Wood2010, Papamakarios2016,
  Papamakarios2019, Chen2019, Thomas2020, Hermans2020}. However, both
OMC and robust OMC, which we introduce next, are based on the
approximation in \eqref{eq:approx_likelihood}.


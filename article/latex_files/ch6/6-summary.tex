In this paper, we presented the implementation details we followed for
developing the LFI method ROMC at the \pkg{ELFI} package. We paid
thorough attention to two specific use-case scenarios. Firstly, we
illustrate how a user may take advantage of our ready-to-use API for
solving its LFI problem. Secondly, we focus on the scenario where a
researcher wants to intervene and alter parts of the method. Our
implementation is designed to support this as well.

There are still open challenges for the left for future research. Two
directions may enable ROMC to solve high-dimensional problems
efficiently. The first one is enabling ROMC's execution into a cluster
of computers. ROMC can be characterized as an \textit{embarrassingly
  parallel} workload; each optimization problem is an entirely
independent task. Therefore, supporting inference into a cluster of
computers can radically speed up the inference. The second one refers
to implementing the method in a framework that supports automatic
differentiation. Automatic differentiation is necessary for
efficiently solving optimisation problems, especially in
high-dimensional parametric models.

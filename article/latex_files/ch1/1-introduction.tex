Simulator-based models are particularly captivating due to the
modelling freedom they provide. In essence, a simulator-based model
can describe any data generating mechanism that can be written as a
finite set of algorithmic steps. Such freedom comes at a cost;
performing the inference, i.e., sampling or evaluating the posterior,
is challenging.

Optimisation Monte Carlo (OMC) proposed by~\citet{Meeds2015} is a
novel LFI approach for approximating the posterior distribution. The
central idea is turning the stochastic data-generating mechanism into
a set of deterministic optimisation processes. Afterwards,
\citet{Forneron2016} provided a similar method under the name `reverse
sampler'. In their work,~\citet{Ikonomov2019}, located some critical
limitations of OMC. They proposed Robust OMC (ROMC), an alternative
version of OMC with appropriate modifications.

In this paper, we present an implementation of ROMC at the
\proglang{Python} package \pkg{Engine for Likelihood-Free inference
(ELFI)}. We follow appropriate design principles for ensuring
extensibility. As we describe analytically in the next chapter, ROMC
is a general framework for obtaining weighted samples from the
posterior; it defines a sequence of algorithmic steps without
enforcing a specific algorithm for solving each step\footnote{For
being a ready-to-use algorithm,~\citet{Ikonomov2019} proposed a
default method for each step, but this choice is by no means
restrictive.}. A researcher may adopt ROMC and develop novel methods
to solve each task. We have designed our software for facilitating
such alterations. Finally, we have tested the accuracy and the
efficiency of our implementation on some LFI examples supported by the
\pkg{ELFI} package.

To the best of our knowledge, this is the first implementation of the
ROMC inference method to a generic LFI framework, so there are no
other packages for direct comparisons. Therefore, we test ROMC against
(i) an artificial example with tractable likelihood and (ii) the
second-order moving average (MA2) example from the \pkg{ELFI}
package. In the latter, we generate data using known parameters and
compare our results with the typical Rejection
ABC~\citet{lintusaari2017}.
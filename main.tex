\documentclass[article]{jss}

%% -- LaTeX packages and custom commands ---------------------------------------

%% recommended packages
\usepackage{thumbpdf,lmodern}

%% another package (only for this demo article)
\usepackage{framed}

%% new custom commands
\newcommand{\class}[1]{`\code{#1}'}
\newcommand{\fct}[1]{\code{#1()}}


%% -- Article metainformation (author, title, ...) -----------------------------

%% - \author{} with primary affiliation
%% - \Plainauthor{} without affiliations
%% - Separate authors by \And or \AND (in \author) or by comma (in \Plainauthor).
%% - \AND starts a new line, \And does not.
\author{Achim Zeileis\\Universit\"at Innsbruck
   \And Second Author\\Plus Affiliation}
\Plainauthor{Achim Zeileis, Second Author}

%% - \title{} in title case
%% - \Plaintitle{} without LaTeX markup (if any)
%% - \Shorttitle{} with LaTeX markup (if any), used as running title
\title{An Extendable \proglang{Python} Implementation of Robust Optimisation Monte Carlo}
\Plaintitle{An Extendable Python Implementation of ROMC}
\Shorttitle{An Extendable \proglang{Python} Implementation of ROMC}

%% - \Abstract{} almost as usual
\Abstract{
  This short article illustrates how to write a manuscript for the
  \emph{Journal of Statistical Software} (JSS) using its {\LaTeX} style files.
  Generally, we ask to follow JSS's style guide and FAQs precisely. Also,
  it is recommended to keep the {\LaTeX} code as simple as possible,
  i.e., avoid inclusion of packages/commands that are not necessary.
  For outlining the typical structure of a JSS article some brief text snippets
  are employed that have been inspired by \cite{Zeileis+Kleiber+Jackman:2008},
  discussing count data regression in \proglang{R}. Editorial comments and
  instructions are marked by vertical bars.
}

%% - \Keywords{} with LaTeX markup, at least one required
%% - \Plainkeywords{} without LaTeX markup (if necessary)
%% - Should be comma-separated and in sentence case.
\Keywords{JSS, style guide, comma-separated, not capitalized, \proglang{R}}
\Plainkeywords{JSS, style guide, comma-separated, not capitalized, R}

%% - \Address{} of at least one author
%% - May contain multiple affiliations for each author
%%   (in extra lines, separated by \emph{and}\\).
%% - May contain multiple authors for the same affiliation
%%   (in the same first line, separated by comma).
\Address{
  Achim Zeileis\\
  Journal of Statistical Software\\
  \emph{and}\\
  Department of Statistics\\
  Faculty of Economics and Statistics\\
  Universit\"at Innsbruck\\
  Universit\"atsstr.~15\\
  6020 Innsbruck, Austria\\
  E-mail: \email{Achim.Zeileis@R-project.org}\\
  URL: \url{https://eeecon.uibk.ac.at/~zeileis/}
}

\begin{document}

%% See jss_template_article.tex for formatting instructions and macros to e.g. typset code
%% I only kept the informative sidebars as comments

%% -- Introduction -------------------------------------------------------------

%% - In principle "as usual".
%% - But should typically have some discussion of both _software_ and _methods_.
%% - Use \proglang{}, \pkg{}, and \code{} markup throughout the manuscript.
%% - If such markup is in (sub)section titles, a plain text version has to be
%%   added as well.
%% - All software mentioned should be properly \cite-d.
%% - All abbreviations should be introduced.
%% - Unless the expansions of abbreviations are proper names (like "Journal
%%   of Statistical Software" above) they should be in sentence case (like
%%   "generalized linear models" below).

% \section[Introduction: Count data regression in R]{Introduction: Count data regression in \proglang{R}} \label{sec:intro}
\section{Introduction} \label{sec:intro}

% @Vasilis: we can work on the intro when we have the basic structure of the article in place.


% \begin{leftbar}
% The introduction is in principle ``as usual''. However, it should usually embed
% both the implemented \emph{methods} and the \emph{software} into the respective
% relevant literature. For the latter both competing and complementary software
% should be discussed (within the same software environment and beyond), bringing
% out relative (dis)advantages. All software mentioned should be properly
% \verb|\cite{}|d. (See also Appendix~\ref{app:bibtex} for more details on
% \textsc{Bib}{\TeX}.)

% For writing about software JSS requires authors to use the markup
% \verb|\proglang{}| (programming languages and large programmable systems),
% \verb|\pkg{}| (software packages), \verb|\code{}| (functions, commands,
% arguments, etc.). If there is such markup in (sub)section titles (as above), a
% plain text version has to be provided in the {\LaTeX} command as well. Below we
% also illustrate how abbrevations should be introduced and citation commands can
% be employed. See the {\LaTeX} code for more details.
% \end{leftbar}



%% -- Manuscript ---------------------------------------------------------------

%% - In principle "as usual" again.
%% - When using equations (e.g., {equation}, {eqnarray}, {align}, etc.
%%   avoid empty lines before and after the equation (which would signal a new
%%   paragraph.
%% - When describing longer chunks of code that are _not_ meant for execution
%%   (e.g., a function synopsis or list of arguments), the environment {Code}
%%   is recommended. Alternatively, a plain {verbatim} can also be used.
%%   (For executed code see the next section.)


% \begin{leftbar}
% As the synopsis above is a code listing that is not meant to be executed,
% one can use either the dedicated \verb|{Code}| environment or a simple
% \verb|{verbatim}| environment for this. Again, spaces before and after should be
% avoided.

% Finally, there might be a reference to a \verb|{table}| such as
% Table~\ref{tab:overview}. Usually, these are placed at the top of the page
% (\verb|[t!]|), centered (\verb|\centering|), with a caption below the table,
% column headers and captions in sentence style, and if possible avoiding vertical
% lines.
% \end{leftbar}

%% -- Illustrations ------------------------------------------------------------

%% - Virtually all JSS manuscripts list source code along with the generated
%%   output. The style files provide dedicated environments for this.
%% - In R, the environments {Sinput} and {Soutput} - as produced by Sweave() or
%%   or knitr using the render_sweave() hook - are used (without the need to
%%   load Sweave.sty).
%% - Equivalently, {CodeInput} and {CodeOutput} can be used.
%% - The code input should use "the usual" command prompt in the respective
%%   software system.
%% - For R code, the prompt "R> " should be used with "+  " as the
%%   continuation prompt.
%% - Comments within the code chunks should be avoided - these should be made
%%   within the regular LaTeX text.

% \begin{leftbar}
% For code input and output, the style files provide dedicated environments.
% Either the ``agnostic'' \verb|{CodeInput}| and \verb|{CodeOutput}| can be used
% or, equivalently, the environments \verb|{Sinput}| and \verb|{Soutput}| as
% produced by \fct{Sweave} or \pkg{knitr} when using the \code{render_sweave()}
% hook. Please make sure that all code is properly spaced, e.g., using
% \code{y = a + b * x} and \emph{not} \code{y=a+b*x}. Moreover, code input should
% use ``the usual'' command prompt in the respective software system. For
% \proglang{R} code, the prompt \code{"R> "} should be used with \code{"+  "} as
% the continuation prompt. Generally, comments within the code chunks should be
% avoided -- and made in the regular {\LaTeX} text instead. Finally, empty lines
% before and after code input/output should be avoided (see above).
% \end{leftbar}

\section{Background}

\subsection{Likelihood-free inference}

%% Material from sec 1.1, 2.1 

\subsection{Robust optimisation Monte Carlo (ROMC)}
%% section 2.2 and 2.3 (alternatively, we could have 2.3 only in the next section, see below)

\subsection{Engine for likelihood-free inference (ELFI)}
%% 2.4

\section{ROMC module in ELFI}  % better title needed

%%  alternatively, we could have here a subsection called "Design principles" with 2.3 and some material from 3.1.1
\subsection{For the user}
%% 3.1

\subsection{For the developer}
%% 3.2

\section{Example}
% perhaps just the MA example, with a subset of the timing plots

%% -- Summary/conclusions/discussion -------------------------------------------

\section{Summary and discussion} \label{sec:summary}


%% -- Optional special unnumbered sections -------------------------------------

% \section*{Computational details}

% \begin{leftbar}
% If necessary or useful, information about certain computational details
% such as version numbers, operating systems, or compilers could be included
% in an unnumbered section. Also, auxiliary packages (say, for visualizations,
% maps, tables, \dots) that are not cited in the main text can be credited here.
% \end{leftbar}

% The results in this paper were obtained using
% \proglang{R}~3.4.1 with the
% \pkg{MASS}~7.3.47 package. \proglang{R} itself
% and all packages used are available from the Comprehensive
% \proglang{R} Archive Network (CRAN) at
% \url{https://CRAN.R-project.org/}.


\section*{Acknowledgments}

\begin{leftbar}
All acknowledgments (note the AE spelling) should be collected in this
unnumbered section before the references. It may contain the usual information
about funding and feedback from colleagues/reviewers/etc. Furthermore,
information such as relative contributions of the authors may be added here
(if any).
\end{leftbar}


%% -- Bibliography -------------------------------------------------------------
%% - References need to be provided in a .bib BibTeX database.
%% - All references should be made with \cite, \citet, \citep, \citealp etc.
%%   (and never hard-coded). See the FAQ for details.
%% - JSS-specific markup (\proglang, \pkg, \code) should be used in the .bib.
%% - Titles in the .bib should be in title case.
%% - DOIs should be included where available.

\bibliography{refs}


%% -- Appendix (if any) --------------------------------------------------------
%% - After the bibliography with page break.
%% - With proper section titles and _not_ just "Appendix".

% \newpage

% \begin{appendix}

% \section{More technical details} \label{app:technical}

% \begin{leftbar}
% Appendices can be included after the bibliography (with a page break). Each
% section within the appendix should have a proper section title (rather than
% just \emph{Appendix}).

% For more technical style details, please check out JSS's style FAQ at
% \url{https://www.jstatsoft.org/pages/view/style#frequently-asked-questions}
% which includes the following topics:
% \begin{itemize}
%   \item Title vs.\ sentence case.
%   \item Graphics formatting.
%   \item Naming conventions.
%   \item Turning JSS manuscripts into \proglang{R} package vignettes.
%   \item Trouble shooting.
%   \item Many other potentially helpful details\dots
% \end{itemize}
% \end{leftbar}


% \section[Using BibTeX]{Using \textsc{Bib}{\TeX}} \label{app:bibtex}

% \begin{leftbar}
% References need to be provided in a \textsc{Bib}{\TeX} file (\code{.bib}). All
% references should be made with \verb|\cite|, \verb|\citet|, \verb|\citep|,
% \verb|\citealp| etc.\ (and never hard-coded). This commands yield different
% formats of author-year citations and allow to include additional details (e.g.,
% pages, chapters, \dots) in brackets. In case you are not familiar with these
% commands see the JSS style FAQ for details.

% Cleaning up \textsc{Bib}{\TeX} files is a somewhat tedious task -- especially
% when acquiring the entries automatically from mixed online sources. However,
% it is important that informations are complete and presented in a consistent
% style to avoid confusions. JSS requires the following format.
% \begin{itemize}
%   \item JSS-specific markup (\verb|\proglang|, \verb|\pkg|, \verb|\code|) should
%     be used in the references.
%   \item Titles should be in title case.
%   \item Journal titles should not be abbreviated and in title case.
%   \item DOIs should be included where available.
%   \item Software should be properly cited as well. For \proglang{R} packages
%     \code{citation("pkgname")} typically provides a good starting point.
% \end{itemize}
% \end{leftbar}

% \end{appendix}

%% -----------------------------------------------------------------------------


\end{document}
